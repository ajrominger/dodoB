\documentclass[11pt]{article}
\usepackage[margin=1in]{geometry}
\geometry{letterpaper}
\usepackage{graphicx}
\usepackage{setspace}
\usepackage{amssymb}
\usepackage{amsmath}
\usepackage{epstopdf}
\usepackage{xcolor}
\usepackage{colortbl}
\usepackage{array}
\usepackage[numbers,sort&compress]{natbib}
\usepackage{fancyhdr}
% \pagestyle{fancy}
% \fancyhead{}
% \fancyhead[LO,LE]{Rominger, {\it et al.}}
% \fancyhead[RO,RE]{Student Training Plan}

\usepackage[T1]{fontenc}
\usepackage{titling}
\setlength{\droptitle}{-3.5em}

\usepackage{wrapfig}

\usepackage{multirow}

%% make lists more compact
\usepackage{enumitem}
\setitemize{noitemsep,topsep=0pt,parsep=0pt,partopsep=0pt}
\setenumerate{noitemsep,topsep=0pt,parsep=0pt,partopsep=0pt}


%% make sections, subsections and paragraphs more compact
\makeatletter
\renewcommand\section{\@startsection{section}{1}{\z@}%
                                  {-1.5ex \@plus -1ex \@minus 0.2ex}%
                                  {0.1ex \@plus 0.2ex}%
                                  {\normalfont\Large\bfseries}}
\makeatother

\makeatletter
\renewcommand\subsection{\@startsection{subsection}{1}{\z@}%
                                  {-1.5ex \@plus -1ex \@minus 0.2ex}%
                                  {0.1ex \@plus 0.2ex}%
                                  {\normalfont\large\bfseries}}
\makeatletter

\makeatletter
\renewcommand\subsubsection{\@startsection{subsection}{1}{\z@}%
                                  {-1.5ex \@plus -1ex \@minus 0.2ex}%
                                  {0.1ex \@plus 0.2ex}%
                                  {\normalfont\bfseries}}
\makeatletter

\makeatletter
\renewcommand{\paragraph}{\@startsection{paragraph}{4}{\z@}
  {1.5ex \@plus 1ex \@minus .2ex}{-1em}
  {\normalfont\normalsize\it}
}
\makeatother


\title{Student Training Plan \vspace{-1.5ex}}
\author{}
\date{}


\begin{document}
\maketitle
% \thispagestyle{fancy} 
\vspace{-6em}



%% how will the tools we give them, give them jobs

Training on this project is focused on closing training gaps
specifically centering around team science and quantitatively
intensive research.
%% what are these gaps: "enhance the future work force"
Enhancing the future work force depends critically on filling these
gaps \citep{woodin2010}. Several of the co-PIs have long been engaged
in bringing biodiversity and data science science to the undergraduate
and graduate levels \citep[e.g.,][]{cook2014, lacey2017}. Co-PIs
Guralnick, Soltis, and Carnaval are on the Steering Committee of a new
RCN entitled, ``Biodiversity Literacy in Undergraduate Education
(BLUE)'' (A. Monfils, PI), and Soltis and Guralnick have also hosted
$>10$ data science workshops in partnership with the Carpentries at
the University of Florida (UF). 

We will focus on hands-on workshops and cross-training between labs to
fill team science and data science knowledge gaps. This collaborative,
cross-institution plan stresses the importance of broad training in
biodiversity science as an integrative discipline and how new tools
are critical for continuing to assemble the most refined view of the
tree of life, the data layers that adorn it, and other dimensions of
biodiversity.  Our goal is to develop a strong framework for joint
participation in annual meetings, where students can take a lead role
in presenting their efforts and integrating their work with the
efforts of the larger team.  We plan to develop workshops that reach
the broadest community.  The training involvement on this grant
extends to those faculty, staff, and interested members of the
biodiversity community and beyond who want to learn more, as we
describe below. The total package is meant to enhance the ability of
all participants to perform fully interdisciplinary, quantitative, and
integrative work, assets that will help our students and community
secure career opportunities into the future.

\paragraph{Team Science:} Understanding ecological systems at scales
critical for human decision making requires interdisciplinary
scientific synthesis \citep{goring2014}, which is increasingly
performed by teams \citep{wuchty2007}, making effectively contributing
to team science critical for participating in research-based careers. Team
science has fundamentally changed the process of knowledge creation;
teams are more likely than solo authors to produce novel, high-impact
research, and their papers are more frequently cited
\citep{wuchty2007}. The interdisciplinary efforts at the core of
biodiversity science in general and of this project in particular are
critical for scientific discovery. Successful interdisciplinary
training therefore requires training in team science as well as in
scientific disciplines, and a foundational component of the training
program for this project will be integration of team science
principles.  The ``science of team science'' indicates that
high-performing teams are diverse \citep{cheruvelil2014} and
multi-institutional \citep{jones2008}. Critical factors for success
include development of strong interpersonal skills, a shared vision,
strategically identifying team members and building the team, managing
disagreements and conflicts, and setting expectations for sharing
credit and authorship\citep{goring2014, cheruvelil2014}, including
adoption of new metrics for individual success. We will emphasize team
science principles in our individual training of students and postdocs
and through our collective interactions and training as well.

\paragraph{Undergraduate Research Training:} Undergraduates will be
involved in training workshops in data science (detailed below) and
will practice the skills they learn while completing field- and
laboratory-based data collection. Undergraduates will be encouraged to
pursue independent projects focusing on any number of the aspects of
our project, from modeling to data science to empirical ecology and/or
evolution. In the event that extra funding is needed for these
projects, undergraduate-specific opportunities exist at the UC
Berkeley, UC Merced, and City College New York. We will mentor
students in applying for these competitive awards, training them in
critical grant writing skills.

\paragraph{Graduate and Postdoctoral Research Training:} Our main goal
of graduate and postdoctoral activities is to enhance needed
integrative and quantitative training in biodiversity science.
Training at each institution will vary depending on the number of
students/postdocs and the academic level of the trainees. Strong
graduate and postdoctoral involvement is planned for monthly cyber
meetings and all in-person project meetings, including a half-day
symposium for students to present their work and receive feedback from
all members of the team at the working group and workshop to be held
at the Santa Fe Institute. Graduate students supported on this project
will enhance a vital and interactive graduate community and will build
connections via graduate training across institutions in the
project. Graduate students and postdocs will receive training through
their respective labs but will also interact as part of the larger
project, enhancing their skills in team science.  

We also plan for lab rotations so that students and postdocs working
on this project have the opportunity to spend 1-4 week periods at
other institutions, learning about efforts and obtaining necessary
skills.  These cross-training lab rotations will provide the broadest
exposure across the collaboration.  This active training is supported
at all participating graduate institutions by a strong formal
curriculum in data science (through, e.g. UC Berkeley's Information
School and Berkeley Initiative in Data Science; and UF's Informatics
Institute). All participating degree-granting institutions also offer
cutting edge courses in biodiversity science specifically relevant to
our proposed research, including, Advanced Phylogenetics,
Phylogenomics, and statistical population genetic modeling seminars.

\paragraph{Career Development:} Primary aims of any training
experience are to gain new skills, increase scientific independence,
and become competitive on the job market. Student training will
include personalized advice on academic and non-academic career
options based on the interests and long-term career goals of the
students. We will provide training in developing an independent
research program, developing effective collaborations, preparing for
job applications and interviews, combining and integrating teaching
and research, mentoring students at different career stages, and
engaging in broader impact and public outreach activities. We use
several resources to guide our mentoring of both students and
postdocs. We will work with students on various aspects of an academic
career, including manuscript preparation, presentations at
conferences, professional networking, and grant writing. The students
will be integral to the project and will meet with their mentors at
least once per week to discuss progress, problems, and solutions
related to both research and career development.

\paragraph{Community Training:} We plan on one physical workshop (at
UF) and one massively open online course hosted by the Santa Fe
Institute on data skills for biodiversity science to reach broad
audiences.  Newly renovated space at UF and a commitment to community
training make it an ideal location (co-PI Soltis is Director of this
new institute). Given the size of the community to be enabled, we
anticipate 30 participants.

\paragraph{Governance of Training Plan:} The PIs will collectively
assure that the training plan is met.  They will review implementation
of the above plan as part of monthly teleconferences. Criteria by
which we will access the success of our plan include: (1) publication
of student-lead manuscripts, (2) successful procurement of
student-lead funding awards, (3) student presentations at prestigious
conferences, and (4) completion of relevant coursework.


%% suppresses bibliography, then make tex file to compile .bbl in separate doc
\bibliographystyle{grant}
\setbox0\vbox{\bibliography{studentTraining.bib}}

\end{document}



