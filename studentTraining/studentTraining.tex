\documentclass[11pt]{article}
\usepackage[margin=1in]{geometry}
\geometry{letterpaper}
\usepackage{graphicx}
\usepackage{setspace}
\usepackage{amssymb}
\usepackage{amsmath}
\usepackage{epstopdf}
\usepackage{xcolor}
\usepackage{colortbl}
\usepackage{array}
\usepackage[numbers,sort&compress]{natbib}
\usepackage{fancyhdr}
\pagestyle{fancy}
\fancyhead{}
\fancyhead[LO,LE]{Rominger, {\it et al.}}
\fancyhead[RO,RE]{Student Training Plan}

\usepackage[T1]{fontenc}
\usepackage{titling}
\setlength{\droptitle}{-3em}

\usepackage{wrapfig}

\usepackage{multirow}

%% make lists more compact
\usepackage{enumitem}
\setitemize{noitemsep,topsep=0pt,parsep=0pt,partopsep=0pt}
\setenumerate{noitemsep,topsep=0pt,parsep=0pt,partopsep=0pt}


%% make sections, subsections and paragraphs more compact
\makeatletter
\renewcommand\section{\@startsection{section}{1}{\z@}%
                                  {-1.5ex \@plus -1ex \@minus 0.2ex}%
                                  {0.1ex \@plus 0.2ex}%
                                  {\normalfont\Large\bfseries}}
\makeatother

\makeatletter
\renewcommand\subsection{\@startsection{subsection}{1}{\z@}%
                                  {-1.5ex \@plus -1ex \@minus 0.2ex}%
                                  {0.1ex \@plus 0.2ex}%
                                  {\normalfont\large\bfseries}}
\makeatletter

\makeatletter
\renewcommand\subsubsection{\@startsection{subsection}{1}{\z@}%
                                  {-1.5ex \@plus -1ex \@minus 0.2ex}%
                                  {0.1ex \@plus 0.2ex}%
                                  {\normalfont\bfseries}}
\makeatletter

\makeatletter
\renewcommand{\paragraph}{\@startsection{paragraph}{4}{\z@}
  {1.5ex \@plus 1ex \@minus .2ex}{-1em}
  {\normalfont\normalsize\it}
}
\makeatother


\title{Student Training Plan \vspace{-2ex}}
\author{}
\date{}


\begin{document}
\maketitle
\thispagestyle{fancy} 
\vspace{-6em}


Training on this project is focused on closing gaps that have been
recognized across the life science spectrum, from undergraduate
through post-graduate.  The need to fill these gaps is critical
%%%%%%%
(Brewer and Smith/AAAS 2011). Several of the co-PIs have long been
%%%%%%%%
engaged in bringing biodiversity science to the undergraduate level
\citep[e.g.,][]{cook2014, lacey2017}, and co-PIs Guralnick and
Soltis are on the Steering Committee of a new RCN entitled,
``Biodiversity Literacy in Undergraduate Education (BLUE)''
(A. Monfils, PI). At the graduate level, we focus on workshops and
cross-training between labs, stressing the importance of broad
training in biodiversity science as an integrative discipline and how
new tools are critical for continuing to assemble the most refined
view of the tree of life, the data layers that adorn it, and other
dimensions of biodiversity.  At both the graduate and postdoctoral
levels (see also Postdoctoral Mentoring Plan for the latter), our goal
is to develop a strong framework for joint participation in annual
meetings, where students can take a lead role in presenting their
efforts and integrating their work with the efforts of the larger
team.  We plan to develop workshops that reach the broadest communit.
The training involvement on this grant extends to those faculty,
staff, and interested members of the biodiversity community and beyond
who want to learn more, as we describe below. The total package is
meant to enhance the ability of all participants to perform fully
interdisciplinary and integrative work.  We provide more detail on key
training initiatives below.

\paragraph{Team Science:} Understanding ecological systems at scales
critical for human decision making requires interdisciplinary
scientific synthesis \citep{goring2014}, which is increasingly
performed by teams (Wuchty et al. 2007; Smith et al. 2014). Team
science has fundamentally changed the process of knowledge creation;
teams are more likely than solo authors to produce novel, high-impact
research, and their papers are more frequently cited
\citep{wuchty2007}. The interdisciplinary efforts at the core of
biodiversity science in general and of this project in particular are
critical for scientific discovery. Successful interdisciplinary
training therefore requires training in team science as well as in
scientific disciplines, and a foundational component of the training
program for this project will be integration of team science
principles.  The ``science of team science'' indicates that
high-performing teams are diverse \citep{cheruvelil2014} and
multi-institutional \citep{jones2008}. Critical factors for success
include development of strong interpersonal skills, a shared vision,
strategically identifying team members and building the team, managing
disagreements and conflicts, and setting expectations for sharing
credit and authorship\citep{goring2014, cheruvelil2014}, including
adoption of new metrics for individual success. We will emphasize team
science principles in our individual training of students and postdocs
and through our collective interactions and training as well.

\paragraph{Undergraduate Research Training:} Undergraduates will be
involved in training workshops in data science (detailed below) and
will practice the skills they learn while completing field- and
laboratory-based data collection. Undergraduates will be encouraged to
pursue independent projects focusing on any number of the aspects of
our project, from modeling to data science to empirical ecology and/or
evolution. Such projects are supported by funding opportunities for
undergraduate research at all participating institutions, and
complementary courses, both capstone senior seminars in thesis
writing, and analytical training labs.

\paragraph{Graduate and Postdoctoral Research Training:} Our main goal
of graduate and postdoctoral activities is to enhance needed
integrative training in biodiversity science.  Training at each
institution will vary depending on the number of students/postdocs and
the academic level of the trainees. Strong graduate and postdoctoral
involvement is planned for monthly cyber meetings and all in person
project meetings, including a half-day symposium for students to
present their work and receive feedback from all members of the team
at the working group and workshop to be held at the Santa Fe
Institute. Graduate students supported on this project will enhance a
vital and interactive graduate community and will build connections
via graduate training across institutions in the project. Graduate
students and postdocs will receive training through their respective
labs but will also interact as part of the larger project.  We also
plan for lab rotations so that students and postdocs working on this
project have the opportunity to spend 1-4 week periods at other
institutions, learning about efforts and obtaining necessary skills.
These cross-training lab rotations will provide the broadest exposure
across the collaboration. This active training is supported at all
participating graduate institutions by a strong formal curriculum in
biodiversity science, with courses in Principles of Systematic
Biology, Molecular Systematics, Advanced Phylogenetics, Phylogenomics,
and Phylogenetics Seminar.

\paragraph{Career Development:} Primary aims of any training
experience are to gain new skills and increase scientific
independence. Student training will include personalized advice on
academic and non-academic career options based on the interests and
long-term career goals of the students. We will provide training in
developing an independent research program, developing effective
collaborations, preparing for job applications and interviews,
combining and integrating teaching and research, mentoring students at
different career stages, and engaging in broader impact and public
outreach activities. We use several resources to guide our mentoring
of both students and postdocs. We will work with students on various
aspects of an academic career, including manuscript preparation,
presentations at conferences, professional networking, and grant
writing. The students will be integral to the project and will meet
with their mentors at least once per week to discuss progress,
problems, and solutions related to both research and career
development.

\paragraph{Community Training:} We plan on one physical workshop (at
UF) and one massively open online course on data skills for
biodiversity science to reach broad audiences.  Newly renovated space
and a commitment to community training make UF an ideal location
(co-PI Soltis is Director of this new institute). Given the size of
the community to be enabled, we anticipate 30 participants.

\paragraph{Governance of Training Plan:} The PIs will collectively
assure that the training plan is met.  They will review implementation
of the above plan as part of monthly teleconferences as well as via
annual project meetings.


\bibliographystyle{unsrtnat}
\bibliography{studentTraining}

\end{document}



