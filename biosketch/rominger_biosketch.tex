%%%%%%%%%%%%%%%%%%%%%%%%%%%% Document Setup %%%%%%%%%%%%%%%%%%%%%%%%%%%%

\documentclass[10pt]{article}

% This is a helpful package that puts math inside length specifications
\usepackage{calc}

% Layout: Puts the section titles on left side of page
\reversemarginpar

%% Set up for letter-sized paper
\usepackage[paper=letterpaper,
marginparwidth=1.1In,     % Length of section titles
marginparsep=0in,       % Space between titles and text
margin=1in,               % 1 inch margins
includemp]{geometry}

%% Get rid of indenting throughout entire document
\setlength{\parindent}{0in}

%% Make smaller skip between paragraphs
\setlength{\parskip}{-0.1in}

%% Reference the last page in the page number
\usepackage{fancyhdr,lastpage}
\pagestyle{fancy}
\fancyhf{}\renewcommand{\headrulewidth}{0pt}
\fancyfootoffset{\marginparsep+\marginparwidth}
\newlength{\footpageshift}
\setlength{\footpageshift}
{0.5\textwidth+0.5\marginparsep+0.5\marginparwidth-2in}
\lfoot{\hspace{\footpageshift}%
  \parbox{4in}{\, \hfill %
    \arabic{page} of \protect\pageref*{LastPage}
    \hfill \,}}

%% PDF bookmarks
\usepackage{color,hyperref}
\definecolor{darkblue}{rgb}{0.0,0.0,0.3}
\hypersetup{colorlinks,breaklinks,
            linkcolor=darkblue,urlcolor=darkblue,
            anchorcolor=darkblue,citecolor=darkblue}
% for URLs
\usepackage{verbatim}

%%%%%%%%%%%%%%%%%%%%%%%% End Document Setup %%%%%%%%%%%%%%%%%%%%%%%%%%%%


%%%%%%%%%%%%%%%%%%%%%%%%%%% Helper Commands %%%%%%%%%%%%%%%%%%%%%%%%%%%%

% The title (name) with a horizontal rule under it
%
% Usage: \makeheading{name}
%
% Place at top of document. It should be the first thing.

\newcommand{\makeheading}[1]{
  \hspace*{-\marginparsep minus \marginparwidth}%
  \begin{minipage}[t]{\textwidth+\marginparwidth+\marginparsep}%
    \begin{center}
      {\large \bfseries #1}\\[-0.15\baselineskip]%      
      \rule{\columnwidth}{1pt}%
    \end{center}
  \end{minipage}
}

%% The section headings
%% Follow this section IMMEDIATELY with the first line of the section
%% text. Do not put whitespace in between.
\renewcommand{\section}[2]%
        {\pagebreak[3]\vspace{1.3\baselineskip}%
         \phantomsection\addcontentsline{toc}{section}{#1}%
         \hspace{0in}%
         \marginpar{
         \raggedright \scshape #1}#2}

%% tabular styles for including dates in list of CV content
\usepackage{array, xcolor}
\definecolor{lightgray}{gray}{0.8}
\newcolumntype{L}{p{0.19\textwidth}}
\newcolumntype{R}{p{0.81\textwidth}}
\newcommand\VRule{\color{lightgray}\vrule width 0.5pt}

%% To compress itemized environments
\usepackage{enumitem}
\setlist{nolistsep}
\setlist{noitemsep}


%% To add some paragraph space between lines.
%% This also tells LaTeX to preferably break a page on one of these gaps
%% if there is a needed pagebreak nearby.
\newcommand{\blankline}{\quad\pagebreak[3]}
\newcommand{\halfblankline}{\quad\vspace{-0.5\baselineskip}\pagebreak[3]}


%% NOTE: \rcollength is the width of the right column of the table
\newlength{\rcollength}\setlength{\rcollength}{2.75in}%


%% THIS IS THE PRIMARY COMMAND USED FOR ENTERING CV DATA!!!!
\newcommand{\cventry}[2]{%
\begin{tabular}[t]{R !{\VRule} L}
  \begin{minipage}[t][][t]{0.81\textwidth}
    #2
  \end{minipage}
  &
  \parbox[t][][t]{0.19\textwidth}{#1}
\end{tabular}%
\vspace{0.25em}
}


%%%%%%%%%%%%%%%%%%%%%%%% End Helper Commands %%%%%%%%%%%%%%%%%%%%%%%%%%%

%%%%%%%%%%%%%%%%%%%%%%%%% Begin CV Document %%%%%%%%%%%%%%%%%%%%%%%%%%%%

\begin{document}
% \makeauthorbold{Rominger}
\makeheading{Andrew J.~Rominger}

\section{Contact Information}
%
\begin{tabular}[t]{p{\textwidth-\rcollength}p{\rcollength}}
  \multicolumn{2}{l}{Santa Fe Institute} \\  
  1399 Hyde Park Road &\textit{E-mail:} \verb|rominger@santafe.edu|\\
  Santa Fe, New Mexico 87501 USA & \textit{Web:} \verb|ajrominger.github.io|\\
\end{tabular}

\section{(A) Professional Preparation}
%
\cventry{2009}{
  {\bf B.S. Hons. Stanford University}, Stanford, California \\
   Biological Sciences
}
%
\cventry{2016}{
  {\bf Ph.D. University of California}, Berkeley, California \\
  Environmental Science, Policy \& Management
}

\section{(B) Appointments}
%
\cventry{2017}{
  {\bf Omidyar Fellow,} Santa Fe Institute
}
%
\cventry{2016}{
  {\bf Post Doctoral Scholar,} UC Berkeley \\
  Berkeley Initiative in Global Change Biology
}
%
\cventry{2014}{
  {\bf Graduate Student Instructor,} UC Berkeley
}
%
\cventry{2011--2014}{
  {\bf NSF Graduate Research Fellow,} UC Berkeley
}
%
\cventry{2010}{
  {\bf Fulbright Scholar,} Pontificia Universidad Cat\'olica de Chile
}
%
\cventry{2009}{
  {\bf Teaching Assistant,} Stanford University
}


\section{(C) Products (i)}
%
\cventry{2017}{
  \begin{itemize}
  \item[]\hspace{-1.1\leftmargin} 1. \hspace{0.569em} O'Dwyer JP, {\bf Rominger AJ}, Xiao
    X (2017). Reinterpreting Maximum Entropy in Ecology: a null
    hypothesis constrained by ecological mechanism. {\it Ecology
      Letters}. Ecology letters, 20: 832--841.
%
  \item[]\hspace{-1.1\leftmargin} 2. \hspace{1.25em}{\bf Rominger AJ}, Merow C. (2017)
{\tt meteR}: An {\tt R} package for testing the Maximum Entropy Theory
of Ecology. {\it Methods in Ecology and Evolution} 8: 241--247.
\end{itemize}
}
%
\cventry{2016}{
  \begin{itemize}
  \item[]\hspace{-1.1\leftmargin} 3. \hspace{1em}{\bf Rominger AJ}, {\it et
      al.}. (2016). Community assembly on isolated islands: Macroecology meets
    evolution. {\it Global Ecology and Biogeography} 25: 769--780.
  \end{itemize}
}
%
\cventry{2015}{
  \begin{itemize}
  \item[]\hspace{-1.1\leftmargin} 4. \hspace{0.7em} Harte J, {\bf Rominger AJ}, Zhang
    W. (2015). Integrating macroecological metrics and community taxonomic
    structure. {\it Ecology Letters} 18: 1068--1077.
  \end{itemize}
}
%
\cventry{2012}{
  \begin{itemize}
  \item[]\hspace{-1.1\leftmargin} 5. \hspace{0.5em} Karp DS, {\bf Rominger AJ}, Zook J,
    Ranganathan J, Ehrlich PR \& Daily GC. (2012). Intensive
    agriculture erodes $\beta$-diversity
    at large scales. {\it Ecology Letters} 15: 963--970.
  \end{itemize}
}


\section{(C) Products (ii)}
%
\cventry{in press}{
\begin{itemize}
\item[]\hspace{-1.3\leftmargin} 1. \hspace{-3em} Harte J, Newman EA, {\bf Rominger
    AJ}. (2017) Metabolic partition across individuals in ecological
  communities. {\it Global Ecology and Biogeography}. Online early:
  {\tt onlinelibrary.wiley.com/doi/10.1111/geb.12621/full}
  \end{itemize}
}
%
\cventry{2017}{
  \begin{itemize}
  \item[]\hspace{-1.1\leftmargin} 2. \hspace{0.9em} Stegner MA, Karp DS, {\bf Rominger
      AJ}, Hadly EA (2017). Can protected areas really maintain mammalian
    diversity? Insights from a nestedness analysis of the Colorado
    Plateau. {\it Biological Conservation} 209: 546--553.
  \end{itemize}
}
%
\cventry{2013}{
  \begin{itemize}
  \item[]\hspace{-1.2\leftmargin} 3. \hspace{0em} Harte J, Kitzes J, Newman E, {\bf
      Rominger AJ}. (2013). Taxon categories and the universal
    species-area relationship: A comment on Sizling et al.  {\it The
      American Naturalist} 181: 282--287.
  \end{itemize}
}
%
\cventry{2012}{
  \begin{itemize}
  \item[]\hspace{-1.1\leftmargin} 4. \hspace{0.8em} Maurer BA, Kembel SW, {\bf Rominger
      AJ} \& McGill BJ. (2012). Estimating metacommunity extent using
    data on species abundances, environmental variation, and
    phylogenetic relationships across geographic space. {\it
      Ecological Informatics} 13: 114--122.
  \end{itemize}
}
%
\cventry{2009}{
  \begin{itemize}
  \item[]\hspace{-1.1\leftmargin} 5. \hspace{1.2em}{\bf Rominger AJ}, Miller TEX \& Collins
    SL. (2009). Relative contributions of neutral and niche-based
    processes to the structure of a desert grassland grasshopper
    community. {\it Oecologia} 161: 791--800.
  \end{itemize}
}

\section{(D) Synergistic Activities}
%
\cventry{2017--present}{
  \begin{itemize}
  \item[]\hspace{-1.1\leftmargin} 1. \hspace{1.25em}{\bf High School Mentor} \\
  Guide high school students in independent projects through the Santa
  Fe Institute's mentoring program.
  \end{itemize}
}

%
\cventry{2013--2016}{
  \begin{itemize}
  \item[]\hspace{-1.1\leftmargin} 1. \hspace{1.25em}{\bf Board Member} \\
  Talking Talons Youth Leadership Community Fund, an organization that
  funds environmental education projects.
  \end{itemize}
}
%
\cventry{2009--present}{
  \begin{itemize}
  \item[]\hspace{-1.1\leftmargin} 2. \hspace{1.25em}{\bf Community presentation speaker} \\
  Present at youth and environmental group meetings including Central
  New Mexico Audubon Society and Pacific Internship Programs for
  Exploring Science about science, conservation and environmental
  education.
  \end{itemize}
}
%
\cventry{2009}{
  \begin{itemize}
  \item[]\hspace{-1.1\leftmargin} 3. \hspace{1.25em}{\bf Splash instructor} \\
  Thought an interactive course about global change biology to K-12
  students as part of Stanford University's Educational Studies Splash
  Program.
  \end{itemize}
}
%
\cventry{2006--2011}{
  \begin{itemize}
  \item[]\hspace{-1.1\leftmargin} 4. \hspace{1.25em}{\bf Natural history docent} \\
  Lead classroom and community tours of Jasper Ridge Biological
  Preserve (Stanford University) focusing on local
  conservation issues, ecology, evolution and geology.
  \end{itemize}
}

% \section{Collaborators}
% %
% \begin{tabular}[t]{p{1\textwidth}}
%   E.~Armstrong (U Hawaii); 
%   L.~Becking (Institute for Marine Resources and Ecosystem Studies); 
%   G.~Bennett (U of Hawaii); 
%   M.~Brewer (UC Berkeley);
%   S.~Collins (U New Mexico); 
%   D.~Cotoras (UC Santa Cruz);
%   G.~Daily (Stanford University);
%   R.~Dirzo (Stanford University);
%   P.~Ehrlich (Stanford University);
%   D.~Erwin (Smithsonian);
%   C.~Ewing (U Hawaii);
%   M.~Fuentes (Consejo Nacional de Investigaciones Cient\'ificas y
%   T\'ecnicas);
%   R.~Gillespie (UC Berkeley);
%   K.~Goodman (UC Berkeley);
%   D.~Gruner (U Maryland);
%   E.~Hadly (Stanford University);
%   J.~Harte (UC Berkeley);
%   S.~Kembel (U Qu\'ebec \`a Montr\'eal);
%   J.~Lim (UC Berkeley);
%   P.~Marquet (Pontificia Universidad Cat\'olica de Chile);
%   C.~Marshall (UC Berkeley);
%   N.~Martinez (U Arizona);
%   B.~Mauer (Michigan State University);
%   B.~McGill (U Maine);
%   C.~Merow (U Connecticut);
%   T.~Miller (Rice University);
%   J.~O'Dwyer (U Illinois Urbana-Champaign);
%   P.~O'Grady (UC Berkeley);
%   D.~Percy (Natural History Museum, London);
%   D.~Price (U Hawaii);
%   G.~Roderick (UC Berkeley);
%   K.~Shaw (Cornell University);
%   F.~Valdovinos (U Arizona);
%   P.~Wagner (Smithsonian).
%   X.~Xiao (U Main)
% \end{tabular}



\end{document}
